\documentclass{article}
\usepackage[utf8]{inputenc}

\title{Chapter 1: Basic Concepts}
\author{Exercise Solutions}
\date{}

\begin{document}
\maketitle

\section*{1.1 Difficulty [10]}

\subsection*{Answer}
t \textleftarrow\ a , a \textleftarrow\ b, b \textleftarrow\ c, 
c \textleftarrow\ d, d \textleftarrow\ t
\subsection*{Explanation}
Since we're "overwriting" each value by shifting every variable to the left, 
we have to keep an "extra copy" of \textbf{a} \textit{in} \textbf{t} before it gets overwritten, that way we can
overwrite \textbf{d} \textit{with} \textbf{t} (our "copy" or equivalent to \textbf{a}) at the end- thus
preserving \textbf{a}. Order significantly matters here as we can't go the other direction
 without it being too many steps.
\subsection*{Comments}
This is the equivalent of a nice beginning programming exercise I use
at Girls Who Code! Gets the brain stirring, but definitely a simpler 
exercise that gets the point across.

\end{document}
